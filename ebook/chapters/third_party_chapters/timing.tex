% !TeX root = ../../python-snippets.tex

\section{Timing}

The \textit{Timing} section contains snippets related to timing.
This means, that you'll find ways to time the execution time of certain code parts and much more.
To achieve that, \lstinline{boxx} is mainly used.


\subsection{Timing Tool: boxx}

The third party \lstinline{boxx} module provides you a pretty neat way to time your code executions via \lstinline{with}-blocks.

\lstinputlisting[caption=timing\_tool.py]{../third_party/timing_tool.py}

\begin{lstlisting}[caption=Output of timing\_tool.py]
$ python timing_tool.py
"timeit" spend time: 0.01047492
"sleep" spend time: 0.10027
\end{lstlisting}


\subsection{f-strings VS str}

Let's say you want to convert an integer to a string.
There exist different ways to do that.
Working on an open source project I came across an option I've never thought of: using f-strings.
But which method is faster?
Using the builtin \lstinline{str} function or f-strings?
You can use \lstinline{boxx} to time it easily.

\lstinputlisting[caption=f-strings\_vs\_str.py]{../third_party/f-strings_vs_str.py}

\begin{lstlisting}[caption=Output of f-strings\_vs\_str.py]
$ python f-strings_vs_str.py
"f-strings" spend time: 0.09714508
"str" spend time: 0.1627851
\end{lstlisting}

\textbf{Note:} Even though f-strings are faster in this situation, keep in mind, that the builtin \lstinline{str} function is preferred to be used.
Clean code is usually more important than efficiency.
