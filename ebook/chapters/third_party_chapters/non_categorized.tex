% !TeX root = ../../python-snippets.tex

\section{Non-Categorized}

All recipes, which do not fall into one of the underlying categories, are listed here.

\subsection{Auto Login On Website}

Using \lstinline{selenium} allows you to automatically open a new browser window and login into a certain website, e.g. GitHub.

\lstinputlisting[caption=auto\_login\_website.py]{../third_party/auto_login_website.py}


\subsection{Count Python Bytes In A Directory}

\lstinline{PyFilesystem2} is Python's file system abstraction layer.
You can use a file system object to analyse your files.
The folowing recipe shows you how you can get the number of Python source code bytes in your directory.

\lstinputlisting[caption=count\_python\_bytes.py]{../third_party/count_python_bytes.py}


\subsection{Create World Maps}

\lstinline{folium} builds on the data wrangling strengths of the Python ecosystem and the mapping strengths of the Leaflet.js library.
Manipulate your data in Python, then visualize it in a Leaflet map via \lstinline{folium}.
This recipe creates a world map with the USA in the centre of the map.

\lstinputlisting[caption=folium\_snippet.py]{../third_party/folium_snippet.py}


\subsection{Print Formatted JSON}

Printing JSON object formatted is as easy as:

\lstinputlisting[caption=formatted\_json.py]{../third_party/formatted_json.py}

\begin{lstlisting}[caption=Output of formatted\_json.py]
$ python formatted_json.py
{
    "userId": 1,
    "id": 1,
    "title": "delectus aut autem",
    "completed": false
}
\end{lstlisting}

\textbf{Note:} The snippet is part of the third party part as it makes use of the \lstinline{requests} library.
The \lstinline{json} module is part of Pythons standard library.


\subsection{Inspect Docker}

You can inspect running Docker containers and existing images using the \lstinline{docker} module.

\lstinputlisting[caption=inspect\_docker.py]{../third_party/inspect_docker.py}


\subsection{Is Holiday}

The \lstinline{holidays} module provides you an elegant and easy way to check, whether a given date is a holiday in the specified region.

\lstinputlisting[caption=is\_holiday.py]{../third_party/is_holiday.py}


\subsection{Web Scraping}

The recipe in the \lstinline{mathematicians.py} file shows you how you can scrape information from the internet.
It makes use of the following third party packages:

\begin{itemize}
    \item BeautifulSoup
    \item requests
\end{itemize}

The recipe returns you the most popular mathematicians.
As it's to long, please have a look at it in the repository.


\subsection{Interacting With The Medium API}

Another recipe, which is to long but worth to mention, is the one contianed by the \lstinline{medium.py} file.
Running the file gives you the ability to interact with the Medium API via the command-line.


\subsection{Mocking Requests}

When writing tests for your application, you may come across the situation, where you have to mock requests.
The following Listing shows you how you can do this by using the standard libraries \lstinline{unittest.mock} module.

\lstinputlisting[caption=mocking\_requests.py]{../third_party/mocking_requests.py}


\subsection{Mypy Example}

The following Listing reveals the usage of \lstinline{mypy} as a static type checker.
Running the snippet will throw a \lstinline{TypeError} as expected.

\lstinputlisting[caption=mypy\_example.py]{../third_party/mypy_example.py}


\subsection{NumPy Array Operations}

Running the following code snippet reveals you some of the existing \lstinline{numpy} array operations.

\lstinputlisting[caption=numpy\_array\_operations.py]{../third_party/numpy_array_operations.py}

\begin{lstlisting}[caption=Output of numpy\_array\_operations.py]
$ python numpy_array_operations.py
[ 2  4  6  8 10]
[ 1  4  9 16 25]
\end{lstlisting}


\subsection{Parse Complex Excel Sheets}

You can parse more complex Excel Sheets by using \lstinline{pandas} and installed a \lstinline{pandas} extension called \lstinline{xlrd}.

\lstinputlisting[caption=parse\_complex\_excel\_sheets.py]{../third_party/parse_complex_excel_sheets.py}


\subsection{Nmap For Python}

Creating your own port scanner becomes easy when using \lstinline{nmap}.
Luckily, a Python package exists providing access to nmap: \lstinline{python-nmap}.
\textbf{Hint:} Make sure you have Nmap installed on your operating system as \lstinline{python-nmap} only provides access to the Nmap API.

\lstinputlisting[caption=port\_scanner\_nmap.py]{../third_party/port_scanner_nmap.py}


\subsection{Test Renamed Class}

Renaming a class can happen over time.
If you want to test, wether your code is backwards compatible, you can make use of the following snippet.

\lstinputlisting[caption=pytest\_rename\_class\_backwards\_compatibility.py]{../third_party/pytest_rename_class_backwards_compatibility.py}


\subsection{Reduce Pandas Dataframe Memory}

When dealing with large data sets, it can be an advantage to change column types from \lstinline{object} to \lstinline{category} as shown in the next Listing.

\lstinputlisting[caption=reduce\_pandas\_df\_memory.py]{../third_party/reduce_pandas_df_memory.py}


\subsection{Async Libraries}

Before \lstinline{async}/\lstinline{await} became so popular and part of the standard library, you had several options to use asynchronous techniques in your project.
The three main ones were \lstinline{asynio}, \lstinline{gevent} and \lstinline{tornado}.
You can find an example for each in the \textit{third\_party} directory of the repository.
The file names are as follows:

\begin{itemize}
    \item \lstinline{test_asyncio.py}
    \item \lstinline{test_gevent.py}
    \item \lstinline{test_tornado.py}
\end{itemize}


\subsection{Unzip World Bank Data}

The following recipe shows you how you can download a \lstinline{.zip} file from an online source, extract and work with the data.
Therefore, except the \lstinline{requests} library only standard library modules are used.

\lstinputlisting[caption=world\_bank\_data.py]{../third_party/world_bank_data.py}


\subsection{Simple Debugger}

When writing Python code you may come across situations, where you want to find out, what's going on.
Sometimes you may make use of an actual debugger, but most of the times inserting some \lstinline{print}-statements seems just fine.
The \lstinline{PySnooper} package is probably the best reason, why you should never use \lstinline{print}-statements to debug your code again.
The following Listing shows you the simple usage of its \lstinline{snoop} decorator.

\lstinputlisting[caption=simple\_debugger.py]{../third_party/simple_debugger.py}

The output is too long to be covered here.
Feel free to run the snippet from your terminal and get a sense of how amazing and helpful this little package can be.


\subsection{Text Analysis}

If you want a simple way to perform text analysis, you can use \lstinline{TextBlob}.
The following snippet shows you a sample usage.
Make sure to read the docs of the package as it provides functionality for translations and further analysis, too.

\lstinputlisting[caption=text\_analysis.py]{../third_party/text_analysis.py}

\textbf{Note:} Make sure to download the needed data before hand, otherwise an exception is thrown telling you to download it.

\begin{lstlisting}[caption=Download data using NLTK]
>>> import nltk
>>> nltk.download('averaged_perceptron_tagger')
\end{lstlisting}


\subsection{Complex List Ops}

In your day to day work you will very likely come across performance issues in Python.
Sometimes it's a good idea to choose a different data structure than you have used to speed up certain operations.
However, you may come across situations where this is even not enough.
Instead of switching to another language like C, you may want to check out the \lstinline{blist} package first.
\glqq The \lstinline{blist} is a drop-in replacement for the Python list that provides better performance when modifying large lists.\grqq
Here is a small snippet illustrating the huge impact of using a \lstinline{blist} instead of a \lstinline{list}:

\lstinputlisting[caption=complex\_list\_ops.py]{../third_party/complex_list_ops.py}

\begin{lstlisting}[caption=Output of complex\_list\_ops.py]
$ python complex_list_ops.py
"Builtin" spend time: 24.61005
"Blist" spend time: 4.675813
\end{lstlisting}

\glqq The blist package also provides \lstinline{sortedlist}, \lstinline{sortedset}, [...] \lstinline{btuple} types\grqq , and much more.
