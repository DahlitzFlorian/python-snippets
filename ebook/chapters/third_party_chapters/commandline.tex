% !TeX root = ../../python-snippets.tex

\section{Command-Line}

In this section you will find useful snippets when dealing with command-lines.

\subsection{Attention Please!}

Sometimes you need to create an eye catcher on the command-line.
Therefore, you can use \lstinline{termcolor} to create one.
In the following code snippet you will find the code necessary to create a blinking white message on a red background.

\lstinputlisting[caption=attention\_message.py]{../third_party/attention_message.py}


\subsection{Colored Python}

As indicated in the previous recipe, you can use \lstinline{termcolor} to print colored strings on the command-line.

\lstinputlisting[caption=colored\_python.py]{../third_party/colored_python.py}


\subsection{Generate CLI Help Strings}

Making use of \lstinline{docopt} you can generate cli help strings based on Python docstrings.

\lstinputlisting[caption=cli\_help\_strings.py]{../third_party/cli_help_strings.py}

\begin{lstlisting}[caption=Output of cli\_help\_string.py]
$ python cli_help_strings.py --help
CLI HELP STRINGS
Usage:
    cli_help_strings.py
    cli_help_strings.py <name>
    cli_help_strings.py -h|--help
    cli_help_strings.py -v|--version
Options:
    <name>  Optional name argument.
    -h --help  Show this screen.
    -v --version  Show version.
\end{lstlisting}


\subsection{Parse And Print Colored Arguments}

\lstinline{clint} provides you the functionalities to parse command-line arguments and to print them in a colored way.

\lstinputlisting[caption=clint\_cli\_tool.py]{../third_party/clint_cli_tool.py}

\begin{lstlisting}[caption=Output of clint\_cli\_tool.py]
$ python clint_cli_tool.py
>>> Arguments passed in: []
>>> Flags detected: <args []>
>>> Files detected: []
>>> NOT Files detected: <args []>
>>> Grouped Arguments: {'_': <args []>}
\end{lstlisting}

\textbf{Note:} The keywords between \lstinline{>>>} and the colon are blue.


\subsection{Print Tables}

You can use \lstinline{prettytable} to print a table on the command-line.

\lstinputlisting[caption=display\_tables.py]{../third_party/display_tables.py}

\begin{lstlisting}[caption=Output of display\_tables.py]
$ python display_tables.py
+------+-------+
| food | price |
+------+-------+
| ham  |   $2  |
| eggs |   $1  |
| spam |   $4  |
+------+-------+
\end{lstlisting}


\subsection{Fancy CLI Header}

Ever wondered how other projects create those fancy cli headers?
You can use \lstinline{pyfiglet} to achieve exactly that!

\lstinputlisting[caption=fancy\_cli\_header.py]{../third_party/fancy_cli_header.py}

The following output is not included as image.
It's copied and pasted directly from the command-line.
So if you recognize a non-fancy header, make sure to try it yourself to get a better impression of what you can achieve with this recipe.

\begin{minipage}{\textwidth}
\begin{lstlisting}[caption=Output of fancy\_cli\_header.py]
    __            __     __       
    / /____  _  __/ /_   / /_____  
   / __/ _ \| |/_/ __/  / __/ __ \ 
  / /_/  __/>  </ /_   / /_/ /_/ / 
  \__/\___/_/|_|\__/   \__/\____/     
                                                                     
\end{lstlisting}
\end{minipage}


\subsection{Interactive CLI}

By using \lstinline{PyInquirer} you can create interactive command-line interfaces.
As the file is to long to be displayed here, I refer to the \lstinline{interactive_cli.py} file contained by the \textit{third\_party} directory of the repository.
I recommend to run the snippet and get a feeling for what's possible with this amazing package.


\subsection{Creating A Progress Bar}

As the name already suggests, \lstinline{progressbar} can be used to create a progress bar on the command-line.

\lstinputlisting[caption=shows\_progress.py]{../third_party/show_progress.py}


\subsection{Creating A Simple Progress Bar}

This code snippet is similar to the last one except the fact, that this one uses the \lstinline{tqdm} package.

\lstinputlisting[caption=simple\_progressbar.py]{../third_party/simple_progressbar.py}
