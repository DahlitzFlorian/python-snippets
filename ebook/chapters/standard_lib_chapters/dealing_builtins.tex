% !TeX root = ../../python-snippets.tex

\section{Dealing with builtins}

Sometimes it can be helpful to manipulate builtins or to find a better way to deal with them.
This section will show you ways to interact with builtins you may not thought of.


\subsection{Change Behaviour}

In some situations it can be helpful to extend the functionality of a certain builtin.
Keep in mind, that this can be dangerous if it's not documented and is done in a global scope!

\lstinputlisting[caption=builtins\_manipulation.py]{../standard_lib/builtins_manipulation.py}

The Listing first assigns the builtin \lstinline{print} function to a variable called \lstinline{_print}.
After that, a custom \mbox{\lstinline{print}} function is defined shadowing the builtin one.
The functionality is extended by displaying the number of arguments before printing the actual output.
You can find the output in the following Listing.

\begin{lstlisting}[caption=Output of builtins\_manipulation.py]
$ python builtins_manipulation.py
One Two Three
Number of arguments: 3
\end{lstlisting}


\subsection{Capture Output}

It may be helpful to capture and redirect the output of certain functions.
For instance you don't want to send the output of the builtin \lstinline{help} function to stdout but want to redirect it to a file.
The following Listing shows you three ways how to capture and redirect the output of functions.

\lstinputlisting[caption=capture\_output.py]{../standard_lib/capture_output.py}

The first option saves the output to a \lstinline{StringIO} object.
The value can be accessed using \lstinline{.getvalue()}.

The second option can be used to save the output to a specified file.
In this case we save the output to \lstinline{help.txt}.

Last but not least we are redirecting the output to stderr.
