% !TeX root = ../../python-snippets.tex

\section{Context Manager}

In this section you will find a collection of self-implemented context manager.


\subsection{Open Multiple Files}

Ever wanted to save a certain text to multiple files?
Well, with this context manager you can open multiple files at once and write a specified text to them.

\lstinputlisting[caption=multi\_open\_files.py]{../standard_lib/multi_open_files.py}

If you run the snippet, it will create ten text files all containing the same text.


\subsection{Temporal SQLite Table}

If you are working with \lstinline{sqlite}, you may find this context manager helpful.
It creates a temporal table you can interact with.

\lstinputlisting[caption=temptable\_contextmanager.py]{../standard_lib/temptable_contextmanager.py}

When leaving the \lstinline{with}-statement, the table is deleted.
That said, you can run the snippet as often as you like, the output remains the same.

\begin{lstlisting}[caption=Output of temptable\_contextmanager.py]
$ python temptable_contextmanager.py
(1, 1)
(1, 2)
(2, 1)
(5,)
\end{lstlisting}


\subsection{Timing Context Manager}

Sometimes you simply want to find out, how long a certain code block needs to be executed.
To do that you have different options: Use a third-party package like \lstinline{boxx} or create your own timing context manager based on the standard library.
The following Listing shows you, how you can create your own timing context manager.
In specific, this approach makes use of a generator function (you can implement this context manager based on a class as well).

\lstinputlisting[caption=timing\_context\_manager.py]{../standard_lib/timing_context_manager.py}

As you can see, the code is pretty straightforward.
An example output is shown below.

\begin{lstlisting}[caption=Output of timing\_context\_manager.py]
$ python timing_context_manager.py
Time List Comprehension: 0.5184009075164795
\end{lstlisting}
