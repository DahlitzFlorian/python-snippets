% !TeX root = ../../python-snippets.tex

\section{Files}

In this section you will find a collection of snippets revealing tips for interacting with files.


\subsection{Hash a File}

To ensure a files integrity, you can hash the file and compare it with other hashes.
The following Listing shows you how to hash a certain file using \lstinline{MD5} and \lstinline{SHA1}.
Both hashes are printed to stdout as well as the name of the hashed file.

\lstinputlisting[caption=hash\_file.py]{../standard_lib/hash_file.py}


\subsection{Read Files using Iterator}

Reveals the usage of iterators to read in a file.
It's useful when dealing with large files.
If not using iterators, the whole file is loaded into memory at once (think of several gigabyte huge files).
If using iterators, only the next line is loaded.

\lstinputlisting[caption=read\_files\_using\_iterator.py]{../standard_lib/read_files_using_iterator.py}


\subsection{File Matching Using \lstinline{fnmatch}}

Matching certain strings is easy if you use Pythons built-in \lstinline{fnmatch} module.
It even provides you functionality to translate the easy to use \lstinline{fnmatch} patterns into regular expressions.
The following recipe shows you a sample usage filtering markdown files and \lstinline{git}-related files in your current working directory.

\lstinputlisting[caption=file\_matching\_regex.py]{../standard_lib/file_matching_regex.py}
