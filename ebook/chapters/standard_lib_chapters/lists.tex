% !TeX root = ../../python-snippets.tex

\section{Lists}

Provide useful tips for dealing with lists.


\subsection{Flatten a List}

Sometimes you have a nested list and just want to flatten it.
Here's a small snippet revealing how to achieve right that.

\lstinputlisting[caption=flatten.py]{../standard_lib/flatten.py}

\begin{lstlisting}[caption=Output of flatten.py]
$ python flatten.py
[1, 2, 3, 4, 5, 6]
\end{lstlisting}


\subsection{Priority Queue}

Image you have a sports tournament (e.g. table tennis or basketball).
Now you want to get the player or team with the most points.
You could store all the information in a list, but if new items are added, you need to resort it.
This can take a significant time amount if the data set keeps growing.
You can make use of a \lstinline{Heap} data structure to implement your own priority queue to auto-sort the data for you.

\lstinputlisting[caption=priority\_queue.py]{../standard_lib/priority_queue.py}

As the snippet provides a sample usage, the following Listing shows you the output.

\begin{lstlisting}[caption=Output of priority\_queue.py]
$ python priority_queue.py
hello
None
world
None
\end{lstlisting}


\subsection{Remove Duplicates}

Remove duplicates from a list but keeping the order by using \lstinline{OrderedDict}.

\lstinputlisting[caption=remove\_duplicates\_list.py]{../standard_lib/remove_duplicates_list.py}

\begin{lstlisting}[caption=Output of remove\_duplicates\_list.py]
$ python remove_duplicates_list.py
List with duplicates: ['foo', 'Alice', 'bar', 'foo', 'Bob']
Without duplicates: ['foo', 'Alice', 'bar', 'Bob']
\end{lstlisting}


\subsection{Unpacking Lists Using * Operator}

Assuming you have a list with more elements than you have variables to store the values in.
You can use the \lstinline{*} operator when unpacking a list to store a partial list in one variable.

\lstinputlisting[caption=list\_unpacking.py]{../standard_lib/list_unpacking.py}

\begin{lstlisting}[caption=Output of list\_unpacking.py]
$ python list_unpacking.py
a = 1
b = [2, 3, 4]
c = 5
\end{lstlisting}


\subsection{Remove Elements Not Matching Pattern}

Let's assume you have a list of certain data and want to remove all elements in it, which are not matching a certain pattern.
You could make use of the built-in \lstinline{list.remove()} method, which removes the first match of and then shifts all subsequent data one position to the left.
So, if you want to remove all, you need to loop over it.
However, this approach gives quadratic behaviour.
The following Listing shows you a much better way to achieve this:

\lstinputlisting[caption=remove\_elements\_list.py]{../standard_lib/remove_elements_list.py}

This approach is not only faster, but creates a new, distinct list and then replaces the old contents all at once.
