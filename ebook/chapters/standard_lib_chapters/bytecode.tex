% !TeX root = ../../python-snippets.tex

\section{Bytecode}

This section contains snippets for investigating Pythons bytecode.


\subsection{Disassemble Bytecode - String Conversion}

Ever wanted to know what a string conversion using f-strings or \lstinline{str()} looks like in bytecode?
The following snippet shows you exactly that!

\lstinputlisting[caption=disassemble\_bytecode.py]{../standard_lib/disassemble_bytecode.py}

For displaying the bytecode of a function the builtin \lstinline{dis} module is used.

\begin{lstlisting}[caption=Output of disassemble\_bytecode.py]
$ python disassemble_bytecode.py
============================== f-strings ==============================
  5           0 LOAD_FAST                0 (number)
              2 FORMAT_VALUE             0
              4 RETURN_VALUE
================================ str() ================================
  9           0 LOAD_GLOBAL              0 (str)
              2 LOAD_FAST                0 (number)
              4 CALL_FUNCTION            1
              6 RETURN_VALUE
\end{lstlisting}


\subsection{Human Readable Bytecode}

If you are using Pythons \lstinline{dis} module to get some insights into what's happening behind the scenes, you may like this recipe as it provides a way to print the output in a much more human readable way.

\lstinputlisting[caption=human\_readable\_bytecode.py]{../standard_lib/human_readable_bytecode.py}

\begin{lstlisting}[caption=Output of human\_readable\_bytecode.py]
$ python human_readable_bytecode.py
LOAD_CONST: 0
RETURN_VALUE: 0
\end{lstlisting}

\textbf{Note:} Even though we didn't put a return statement at the end of \lstinline{foo}, Python added it.
That's because every function in Python needs a return statement (specified by the underlying protocols).
